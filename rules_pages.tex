\documentclass[11pt, a5paper, parskip=half-, DIV=12]{scrartcl}

\usepackage{endeavour}
\usepackage{endeavour_book}

\tikzset{starfield/.pic={
%	\node () at (current page.center) {\includegraphics[width=\pagewidth, height=\pageheight]{Images/starfield.png}};
}}

\version{0.2}

\begin{document}
% Colour Cover
\thispagestyle{plain}
%\begin{titlepage}
\AddToShipoutPictureBG{
\begin{tikzpicture}[remember picture, overlay]
	\node () at (current page.center) {\includegraphics[width=\pagewidth, height=\pageheight]{Images/endeavour_cover.png}};
\end{tikzpicture}
}
%\end{titlepage}
{
\colorlet{headfootcolor}{LCARS_ORANGE}%{lightgray!25}
\phantom{a}


\newpage
}

\ClearShipoutPicture
\AddToShipoutPictureBG{
	\begin{tikzpicture}[remember picture, overlay]
	\pic () at (current page.center) {starfield};
		\node[endeavour_box, minimum width=12.6cm, minimum height=18.8 cm] at (current page.center) {};
	\end{tikzpicture}
}

\setcounter{page}{1}
\setmainfont{TeX Gyre Schola}
\normalsize
\raggedright

%\thispagestyle{plain}
\section*{Acknowlegements}
Much of the look and feel of \ENDEAVOUR{} is derived from its art, all of which was created by \textbf{svekloid}. This art was assembled from multiple collections available online at \href{http://shutterstock.com}{shutterstock.com} and then modified by Michael Purcell.  

\subsection*{Playtesters} \label{subsection:playtesters}
The following people helped to create \ENDEAVOUR{} by playing early versions of the game and providing invaluable feedback.\vspace{-1.75ex}
\begin{multicols}{2}
\begin{itemize}[noitemsep]
  \item Keydan Bruce
  \item Dannielle Harden
  \item Andrew Hellyer
%  \item Sarah Hewat
%  \item Scott Joblin
  \item Sen-Foong Lim
  \item David McKenzie
%  \item Holly Moore
  \item Paul Murray
%  \item David Purcell
%  \item Heidi Purcell
  \item Kira Purcell
  \item Luke Purcell
%  \item Meagan Purcell
%  \item Steve Purcell
%  \item Jason Stark
  \item Jo Stephenson
%  \item Pieter Vismans
  \item Brett Witty
\end{itemize}
\end{multicols}

\subsection*{Design Tools} \label{subsection:design-tools}
The following tools were used to create this rule book:
\begin{description}[font=\normalfont\textbullet\space, noitemsep, topsep=-1ex]
	\item[LuaLaTeX:] Typesetting and layout.
	\item[TikZ:] Diagrams and art.
\end{description}
\vspace{1ex}
The fonts used are {\setmainfont{TT Mussels-BoldItalic} TT~Mussels~Bold~Italic},  \textsf{Futura}, and TeX~Gyre~Schola (cf. Century Schoolbook).

\vfill

\begin{tabular}{@{}m{7.775cm}@{\hspace*{0.375cm}}>{\centering\arraybackslash}m{2.6cm}@{}}
\textbf{Contact:} \href{mailto:endeavour.ttrpg@gmail.com}{endeavour.ttrpg@gmail.com}\newline \phantom{This is a test, only a test.} \newline \footnotesize{For use with the \textsc{Paragon} system, ©2020\newline \textbf{John Harper \& Sean Nittner}. \href{http://agon-rpg.com}{AGON-RPG.com}} & \includegraphics[scale=0.175]{Images/paragon_logo_mark.png} \\[5ex]
\footnotesize{This work is licensed under a Creative Commons \newline ``Attribution-ShareAlike 4.0 International'' license.} & \Huge{\doclicenseIcon}
\end{tabular}

\newpage

%\setcounter{page}{2}

\section*{Premise}
The year is 2364.
The people of Earth, together with their network of far-flung colonies, comprise the Terran Civilization. The Terran Civilization, in turn, is a member of the Interstellar Confederation, an organization made up of hundreds of spacefaring civilizations and populated by dozens of different species.

You are an officer aboard the Interstellar Confederation Ship (ICS) Endeavour. Your mission is to explore the galaxy. You will travel deep into uncharted space where you will encounter strange natural phenomena, make first contact with alien civilizations, and help those in need.

\subsection*{Influences}
\ENDEAVOUR{} is inspired by \textsc{Star Trek}. In particular, its design has been informed by the series that Gene Roddenberry was personally involved with: \textsc{Star Trek} (the original series) and \textsc{Star Trek: The Next Generation}.

It has also been informed by other television shows such as \textsc{Quantum Leap}, \textsc{seaQuest DSV}, and \textsc{Stargate SG-1}.

Other major influences include the works of Isaac~Asimov, Arthur~C.~Clarke, Robert~Heinlein, and Carl~Sagan.
 
The ICS Endeavour is named after the HMS Endeavour, the famed British Royal Navy research vessel that Captain Cook commanded during his first voyage of discovery.

\newpage

\section*{Tone}
This is an optimistic-science-fiction game. It is a game about a future in which humanity has progressed beyond the kinds of internecine conflicts that plague modern society. Advanced technology is common and has created a post-scarcity society throughout the Interstellar Confederation.

The crew of the ICS Endeavour are skilled professionals who know how to work together effectively. While they may disagree at times, they will not let personal biases or animosity affect their work.

Stories in \ENDEAVOUR{} generally involve some kind of moral quandary. Moreover, the futuristic setting acts as a lens through which we can view contemporary social issues. The best such stories are fundamentally about the difficult choices the crew are asked to make and how they are affected by their experiences.  

\subsection*{Technology}
\ENDEAVOUR{} is about people more than it is about technology. That said, there are a few technologies that are commonly assumed to be available in order to make stories about deep-space exploration plausible and relatable.
\begin{description}
	\item[Faster Than Light Travel]
	\item[Artificial Gravity]
	\item[Universal Translation]
\end{description}
Without these technologies, or when these technologies fail, stories tend to focus on the consequences of their absence rather than on the characters.

\newpage

\section*{The Interstellar Confederation}
The Interstellar Confederation is a diverse and multicultural society. Its member civilizations lay claim to thousands of star systems and its population numbers in the quadrillions. 

The size and scope of the Interstellar Confederation makes it nearly impossible for any centralized government to manage it all. As such, each member civilization retains a significant amount of autonomy. This has resulted in a complex society with a wide range of social and economic structures.

Member civilizations, however, are united by their belief in the value of education, reason, and individualism. These beliefs are manifest in the Interstellar Confederation's prohibition on interfering with non-spacefaring civilizations and its emphasis on exploration.

\subsection*{Aliens}
Humans are only one of many species that populate the Interstellar Confederation. Many of these species are humanoids. Furthermore, most of these species have similar environmental requirements which allows them to live and work together without the need for adaptive technologies.

Beyond the Interstellar Confederation, however, there is a bewildering array of alien life forms that the crew might encounter in the course of their adventures. These range from simple one-celled organisms to enormous creatures that live their entire lives in the vacuum of space.

\newpage

\section*{The ICS Endeavour}
The ICS Endeavour is a ship designed for deep-space exploration and is capable of operating independently for extended periods of time. By collecting raw materials while underway and using efficient recycling systems, it can produce the air, water, and food needed to sustain the crew for up to one solar year.

The Endeavour has a total complement of about five hundred personnel. This crew is comprised of a multidisciplinary team with expertise in fields such as ship's operations, physical and life sciences, and interstellar diplomacy. It also includes representatives of many species and civilizations.

\subsection*{Captain Nichelle Darcy}
The current captain of the Endeavour is Nichelle Darcy. She is a human from the Terran Civilization and is forty-three solar years old. This is her first command.

Captain Darcy is one of the Interstellar Confederation Fleet's most highly decorated officers. She is a charismatic leader and a cunning tactician. Her past assignments include: executive officer of the ICS Discovery, chief of operations for Habitat Three, and tactical officer aboard the ICS Adventure.

\begin{description}
	\item[Nichelle Darcy (d10):] Captain of the Endeavour (d10), Charismatic (d10), Cunning (d8), Experienced (d8).
\end{description}

In this game, Captain Darcy will be a prominent non-player character. As commanding officer, she will not generally face challenges herself. When she does get personally involved, she will do so by giving a copy of her name die to an officer who used a bond to call upon her while facing a challenge.

\newpage
%\thispagestyle{plain}

\ClearShipoutPicture

\AddToShipoutPictureBG{
\begin{tikzpicture}[remember picture, overlay]
	\pic () at (current page.center) {starfield};

	\node[endeavour_box, minimum width=12.6cm, minimum height=18.8cm,
           path picture={
           	  \path (path picture bounding box.south) --++ (0.5cm,-0.66cm) coordinate (x) {};
               \node at (path picture bounding box.south){
                   \includegraphics[width=12.8cm, height=19cm]{Images/lem_cover.png}
               };
           }] at (current page.center) {};
\end{tikzpicture}
}
\phantom{a}

\newpage

\ClearShipoutPicture
\AddToShipoutPictureBG{
	\begin{tikzpicture}[remember picture, overlay]
		\pic () at (current page.center) {starfield};
		\node[endeavour_box, minimum width=12.6cm, minimum height=18.8 cm] at (current page.center) {};
	\end{tikzpicture}
}


\section*{Systems}
\ENDEAVOUR{} uses a modified version of the \textsc{Paragon System}, found in the \AGON{} roleplaying game. You will need a copy of the \AGON{} rulebook to play this game. Refer to \href{http://www.agon-rpg.com}{agon-rpg.com} for more information. Changes to the \textsc{Paragon System} as used in \ENDEAVOUR{} are described below.

As with \AGON{}, a pool comprised of polyhedral dice is used to determine the outcome of each challenge. It is best if each player has (at least) \textbf{1d4}, \textbf{3d6}, \textbf{3d8}, \textbf{2d10}, and \textbf{1d12} available for use throughout the game.

\subsection*{Modified Terminology}
While \ENDEAVOUR{} uses the same underlying mechanics as \AGON{}, much of the terminology has been changed to reflect the thematic differences between the two games.%
\small
\begin{center}
\begin{tabular}{r@{}c@{}l@{\hskip 4.5ex}r@{}c@{}l} \toprule
\AGON & \tablesep & \ENDEAVOUR & \AGON & \tablesep & \ENDEAVOUR \\ \midrule
Island & \tablesep & Planet & Contest & \tablesep & Challenge \\
& & & Battle & \tablesep & Crisis \\
Gods & \tablesep & Civilizations & Clash & \tablesep & Confrontation \\
Divine Favor & \tablesep & Assistance & & & \\
& & & Pathos & \tablesep & Stress \\
Epithet & \tablesep & Role & Agony & \tablesep & Distress\\
Lineage & \tablesep & Species & Fate & \tablesep & Experience\\ 
Honored God & \tablesep & Heritage & & \\
& & & Glory & \tablesep & Distinction \\
Exodus & \tablesep & Debriefing & & \\
Great Deeds & \tablesep & Discoveries & Harm & \tablesep & Complications \\
Trophies & \tablesep & Artifacts & Perilous & \tablesep & Dangerous \\
& & & Epic & \tablesep & Gruelling \\
Fellowship & \tablesep & Recreation & Sacred & \tablesep & Sensitive\\ 
Sacrifice & \tablesep & Diplomacy & Mythic & \tablesep & Fraught \\ \bottomrule 
\end{tabular}
\end{center}
\normalsize

\newpage

\subsection*{Domains}
Four Domains represent areas of professional expertise for Intersteller Confederation Fleet (ICF) officers. Each challenge falls into one of the four Domains.
\begin{description}
	\item[Leadership \& Negotiation:] Working with others.\\Used for Challenges that require charisma or empathy.
	\item[Science \& Medicine:] Explaining observed phenomena.\\Used for Challenges that require intelligence or creativity.
	\item[Operations \& Engineering:] Managing logistics.\\Used for Challenges that require discipline or precision.
	\item[Strategy \& Tactics:] Outmaneuvering opponents.\\Used for Challenges that require cunning or misdirection.
\end{description}

\subsection*{Virtues}
Four Virtues represent the core values of the Interstellar Confederation. ICF officers are expected to embody these Virtues at all times.
\begin{description}
	\item[Curiosity:] The desire to learn more about the universe.
	\item[Integrity:] Honesty and personal accountability.
	\item[Fairness:] Impartial and just behavior.
	\item[Courage:] The ability to act despite being afraid.
\end{description}
As you play, you may discover that other virtues better describe the core values of the Interstellar Confederation in your game.  If that happens, feel free to replace the existing Virtues with suitable alternatives.

\newpage

\subsection*{Civilizations}
\ENDEAVOUR{} uses Civilizations instead of Gods.  Civilizations include both members of the Interstellar Confederation and any other spacefaring peoples that the crew encounter during their adventures.

Only the Terran Civilization is a part of every game. Other Civilizations will be added as they are encountered during play. Whenever you encounter a new Civilization, you should add it to the list of Civilizations on your character sheet.

\subsection*{Discoveries}
\ENDEAVOUR{} uses Discoveries instead of Great Deeds. Discoveries can take many forms.  Some are academic in nature while other are more personal.

Note that if the crew encounter spacefaring aliens who have their own language and culture, then that should be recorded a new Civilization rather than as a Discovery.

\subsection*{Artifacts}
\ENDEAVOUR{} uses Artifacts instead of Trophies. Artifacts are similar to Discoveries. Most Artifacts are physical objects that either house unusual technology or are imbued with cultural significance.

Note that claiming Artifacts can be problematic. Most will be owned by someone else when they are ``found'' by the crew.  These owners are unlikely to allow the crew to take (or keep) Artifacts without raising some kind of objection.

\newpage

\subsection*{Distinction}
\ENDEAVOUR{} uses Distinction instead of Glory. The way in which an officer earns points of Distinction in \ENDEAVOUR{} differs slightly from the way heroes earn Glory in \AGON{}.

When you are best in a Challenge, you earn one point of Distinction. Each time you earn eight points of Distinction, you advance your Name die.

\subsection*{Complications}
\ENDEAVOUR{} uses Complications instead of Harm.
\begin{description}
	\item[Dangerous:] Mark \tikz[baseline=-0.75ex, scale=0.85, transform shape]{\pic {stress_circle};} (Stress) if you suffer.
	\item[Grueling:] Mark \tikz[baseline=-0.75ex, scale=0.85, transform shape]{\pic {stress_circle};} (Stress) to face the Challenge.
	\item[Sensitive:] Spend \tikz[baseline=-0.75ex]{\pic {earned_divine_favor};} (Assistance) if you suffer.
	\item[Fraught:] Spend \tikz[baseline=-0.75ex]{\pic {earned_divine_favor};} (Assistance) to face the Challenge.
\end{description}

\subsection*{Log Entry}
\ENDEAVOUR{} does not use The Signs of the Gods. Instead, each adventure is prefaced by a Log Entry made by one of the crew. The Log Entry serves to establish the setting and provide background information to the players.

Most often, the Log Entry is taken from the Captain's Log.  This should be prepared in advance and read aloud at the start of the adventure.

The Log Entry should describe where the adventure will take place, what the crew hope to accomplish, and the major characters they expect to encounter.

\newpage

\subsection*{Recreation}
\ENDEAVOUR{} uses Recreation instead of Fellowship. The officers spend time together while off duty. They often pursue mutual hobbies such as playing music or participating in interactive holo-dramas. As they do so, they get to know one another better. 

Taking turns, each player asks a question of another player's character. That player should describe an activity that the two characters do together while off duty and then answer the question. Both players then take a Bond with each other's characters. 

\subsection*{Diplomacy}
\ENDEAVOUR{} uses Diplomacy instead of Sacrifice. The officer who has earned the most points of Distinction leads an effort to strengthen ties between Civilizations. This usually takes the form of informal activities such as athletic competitions, academic conferences, or cultural exchanges.

Diplomacy challenges are always Leadership \& Negotiation challenges.  Everyone who participates marks two \tikz[baseline=-0.75ex]{\pic {earned_divine_favor};}~(Assistance) with Civilizations of their choice. If you are best you also earn a Bond with a Civilization of your choice.

\subsection*{Leadership}
In \ENDEAVOUR{}, the captain is always the leader of the crew. Leadership challenges are used to determine which officer most impresses the captain during each Voyage.

If you are best during a Leadership challenge then you receive a Bond with Captain Darcy. You can use this Bond as usual to ask the captain to Bolster You, Block Complications for You, or Follow Your Lead.

\newpage
%\thispagestyle{plain}

\ClearShipoutPicture

\AddToShipoutPictureBG{
\begin{tikzpicture}[remember picture, overlay]
	\pic () at (current page.center) {starfield};

	\node[endeavour_box, minimum width=12.6cm, minimum height=18.8cm,
           path picture={
           	  \path (path picture bounding box.south) --++ (0.5cm,-0.66cm) coordinate (x) {};
               \node at (path picture bounding box.south){
                   \includegraphics[width=12.8cm, height=19cm]{Images/sunrise_cover.png}
               };
           }] at (current page.center) {};
\end{tikzpicture}
}
\phantom{a}

\newpage

\ClearShipoutPicture

\AddToShipoutPictureBG{
	\begin{tikzpicture}[remember picture, overlay]
		\pic () at (current page.center) {starfield};
		\node[endeavour_box, minimum width=12.6cm, minimum height=18.8 cm] at (current page.center) {};
	\end{tikzpicture}
}


\section*{Character Creation}
Most player characters in \ENDEAVOUR{} are ICF officers who are serving aboard the ICS Endeavour.

To create a character, you will:
\begin{enumerate}
	\item Record your Role. This should describe what you do aboard the ICS Endeavour. Common choices include Executive Officer, Chief Engineer, Helmsperson, etc.  Your Role die begins at \textbf{d6}.
	\item Record your Name. Your Name die begins at \textbf{d6}.
	\item Record your Species. You may be an alien. If you are, describe your Species to the other players.
	\item Record your Heritage. This is your home Civilization. Record two marks of \tikz[baseline=-0.75ex]{\pic {earned_divine_favor};}~(Assistance) with your home Civilization. Record three more marks of \tikz[baseline=-0.75ex]{\pic {earned_divine_favor};}~(Assistance) with Civilizations of your choice.
	\item Choose a favored Domain. Your favored-Domain die begins at \textbf{d8}. Your other three Domain dice begin at \textbf{d6}.
	\item Record two Bonds with each other player character.
\end{enumerate}
You should work together with the other players to create a diverse group of characters.  This will help to ensure that you are prepared to face whatever adventures await you.

\subsection*{Experienced Characters}
Optionally, you can create an experienced character. To do so, follow the process described above. Then, advance your Experience track as far as you like. Take Boons as usual.

Remember that while each Boon will make you a more capable officer, each \tikz[baseline=-0.75ex, scale=0.85, transform shape]{\pic {experience_box};} (Experience) you mark will bring you one step closer to the end of your tour of duty.

\newpage

\ClearShipoutPicture
\AddToShipoutPictureBG{
	\begin{tikzpicture}[remember picture, overlay]
	\pic () at (current page.center) {starfield};
	\end{tikzpicture}
}

{%
\thispagestyle{empty}
\setmainfont{Futura}
\Large
\begin{tikzpicture}[overlay, remember picture]
\pic () at (current page.north west) {character_sheet_1};
\end{tikzpicture}
}

\newpage
{%
\thispagestyle{empty}
\setmainfont{Futura}
\Large
\begin{tikzpicture}[overlay, remember picture]
\pic () at (current page.north west) {character_sheet_2};
\end{tikzpicture}
}

\newpage
%\thispagestyle{plain}

\ClearShipoutPicture

\AddToShipoutPictureBG{
\begin{tikzpicture}[remember picture, overlay]
	\pic () at (current page.center) {starfield};

	\node[endeavour_box, minimum width=12.6cm, minimum height=18.8cm,
           path picture={
           	  \path (path picture bounding box.south) --++ (0.5cm,-0.66cm) coordinate (x) {};
               \node at (path picture bounding box.south){
                   \includegraphics[width=12.8cm, height=19cm]{Images/meteor_shower_cover.png}
               };
           }] at (current page.center) {};
\end{tikzpicture}
}
\phantom{a}

\newpage

\ClearShipoutPicture

\AddToShipoutPictureBG{
	\begin{tikzpicture}[remember picture, overlay]
		\pic () at (current page.center) {starfield};
		\node[endeavour_box, minimum width=12.6cm, minimum height=18.8 cm] at (current page.center) {};
	\end{tikzpicture}
}

\section*{Namarrkon}
\textit{\textbf{Captains Log:} We have arrived in orbit around Namarrkon, a planet perpetually blanketed by ferocious lightning storms. It is the site of an ambitious mining operation that harnesses the energy of the storms to power their equipment.}

\textit{Recent observations have revealed a complex underlying structure in the storms' electrical activity. These patterns suggest that the storms may not be a natural phenomenon.}

\textit{Currently aboard the Endeavour is a team of civilian researchers tasked with studying the storms. Our orders are to transport them to the planet's surface, support their research efforts, and ensure their safety.}

\subsection*{Arrival}
The clouds below roil as the Endeavour orbits the planet. Lightning flashes unceasingly and will make descending to the surface extremely hazardous. 

\textbf{Dr. Stevens} organizes the civilian researchers into small teams and asks that they be distributed at regular intervals across the planet. Doing so will require waiting for a convenient break in the storms and then delivering each team to their destination before the opportunity passes.

\subsubsection*{Deploying the Research Teams}
\begin{itemize}
	\item \textit{Will you proceed with the plan to widely distribute small teams of researchers?} \textbf{Operations \& Engineering} vs. \textbf{Namarrkon}. This is a \textit{Grueling} challenge \textemdash{} Each transfer will be dangerous and time consuming.
	\item \textit{Or will you try to convince Dr. Stevens to change her plan?} \textbf{Leadership \& Negotiation} vs. \textbf{Dr. Stevens}. This is a \textit{Sensitive} challenge \textemdash{} many of the researchers are wary of any outside interference in their work.
\end{itemize}

\newpage

\subsection*{Trials}
\subsubsection*{Culture Clash}
\textbf{Maynard Hill, the mining chief} complains that the researchers are interfering with mining operations. He demands that they be accompanied by his personnel at all times in order to ensure their safety. \textit{Can you convince him to relent?} \textbf{Leadership \& Negotiation} vs. \textbf{Maynard Hill}.

Dr. Stevens complains that the miners are interfering with her research. She demands unrestricted access to the planet. \\ \textit{Will you help the researchers conduct covert operations?} \textbf{Strategy \& Tactics} vs. \textbf{The Miners} (3d6).

\subsubsection*{The Tempest}
A surge in storm activity damages one of the wind farms. Similar surges then damage several other neighboring wind farms. It looks almost like a coordinated attack on the mine's infrastructure. \textit{Will you help the miners repair the damage?} \textbf{Operations \& Engineering} vs. \textbf{Wind Farms} (3d8).

The miners explain that cascade failures are common and caused by how the wind farms are interconnected. \textit{Can you verify whether the miners' explanation is correct?}
\\ \textbf{Science \& Medicine} vs. \textbf{Maynard Hill}.


\subsection*{Crisis}
\begin{itemize}
	\item \textit{Can you find a way to manipulate the storms so that the mine can continue operating?} \textbf{Threats:} The storms attack the mining operation in earnest. Civilian researchers are killed and mining equipment is destroyed. Escape is impossible until the storms recede.
	\item \textit{Or will you try to convince the miners to evacuate?} \\ \textbf{Threats:} The mine cannot be shut down remotely. Miners are caught by the storms. Maynard Hill files a formal complaint with ICF Command.
\end{itemize}

\newpage

\subsection*{Characters}
\begin{description}
	\item[Dr. Stevens (d8):] Lead Civilian Researcher (d10), Creative (d8),  Energetic (d8), Stubborn (d6).
	\item[Maynard Hill (d6):] Chief of Mining Operations (d8), Prickly (d6), Skeptical (d6), Efficient (d8).	
	\item[Namarrkon (d12):] Sentient Storms (d8), Ancient (d8), Unique (d8 \textit{Sensitive}), Vengeful (d8 \textit{Grueling}), \textsc{\textbf{Distributed}} (Cannot be affected by local phenomena).
\end{description}

\subsection*{Places}
Everywhere on Namarrkon is stormy.  Clouds, high winds, and especially lightning are a constant fact of life here.
\begin{description}
	\item[Storm Shelters:] Lightweight and portable. Used as field laboratories by the civilian researchers.
	\item[Wind Farms:] Neatly organized rows of massive wind turbines, all connected by heavy-duty transmission lines.
	\item[Mine Headquarters:] An older building, repeatedly upgraded and reinforced over many years of service.
\end{description}

\subsection*{Mysteries}
\begin{description}
	\item[The storms on Namarrkon are sentient.] \phantom{a} \\ 
	As inorganic entities, the storms are unlike most other known forms of life. \textit{Were the storms created by some lost civilization? Are the storms able to communicate?}
	\item[The mine produces extremely rare minerals.] \phantom{a} \\ These resources are used to create some of the Interstellar Confederation's most fantastic technologies. \textit{Do the storms depend on Namarrkon's unusual geology? Could similar kinds of life be found on other planets?}
\end{description}

\newpage
%\thispagestyle{plain}

\ClearShipoutPicture

\AddToShipoutPictureBG{
\begin{tikzpicture}[remember picture, overlay]
	\pic () at (current page.center) {starfield};

	\node[endeavour_box, minimum width=12.6cm, minimum height=18.8cm,
           path picture={
           	  \path (path picture bounding box.south) --++ (0.5cm,-0.66cm) coordinate (x) {};
               \node at (path picture bounding box.south){
                   \includegraphics[width=12.8cm, height=19cm]{Images/habitat_six_cover.png}
               };
           }] at (current page.center) {};
\end{tikzpicture}
}
\phantom{a}

\newpage

\ClearShipoutPicture
\AddToShipoutPictureBG{
	\begin{tikzpicture}[remember picture, overlay]
		\pic () at (current page.center) {starfield};
		\node[endeavour_box, minimum width=12.6cm, minimum height=18.8 cm] at (current page.center) {};
	\end{tikzpicture}
}

\section*{ICF Habitat Six}
\textit{\textbf{Captain's Log:} We have arrived at ICF Habitat Six. As one of the ICF's first deep-space habitats, this space station is a relic from another age.}

\textit{Years ago, it was decommissioned and ever since has served as an unmanned navigational beacon. Recently, however, the station stopped relaying telemetry. Sensor readings indicate that the habitat's life support systems have been reactivated.}

\textit{We have been unable to establish communications with whoever might be aboard the station. I have dispatched a landing party to investigate.}

\subsection*{Arrival}
When you arrive at the habitat, you are greeted by a small team of \textbf{Bartan} technicians. They explain that they rely on the telemetry produced by the station and came to investigate the cause of its disruption.

``Thank goodness you're here!'' they say, ``We were starting to worry that the ICF had forgotten about this place. There seems to be a problem with the communications array. Can we work together to figure out what's wrong with it?''

\subsubsection*{Covert Communication}
\begin{itemize}[topsep=0ex, partopsep=0ex]
	\item \textit{Will you try to fix the communications array?}\\ \textbf{Operations \& Engineering} vs. \textbf{Habitat Six}.
\end{itemize}
Regardless of the outcome of the challenge, you discover a message hidden in the log files.

\textit{Attention ICF personel: I am a member of the Bartan salvage team currently aboard the station. I have recently begun \textbf{The~Transition} and I do not want to become a King. I am formally requesting political asylum. Please help me.}

\newpage

\subsection*{Trials}
\subsubsection*{Finding the Asylum-Seeker}
\textbf{Surinate}, the member of the salvage team who requested asylum, was careful to conceal his identity. \textit{Can you identify Surinate by analyzing his behavior?}
\textbf{Science \& Medicine} vs. \textbf{Behavioral Cues} (3d6). \textit{Can you convince Surinate to reveal himself?} \textbf{Leadership \& Negotiation} vs. \textbf{Surinate}. 

\subsubsection*{Hostile Reinforcements Arrive}
A Bartan cruiser arrives at the station.  Its captain, \textbf{Taridan}, claims ownership of ICF Habitat Six, citing a treaty that classifies any inoperable and unattended spacecraft as salvage. \textit{Can you convince Taridan that his claim is illegitimate?} \textbf{Leadership \& Negotiation} vs. \textbf{Taridan}.

\subsubsection*{Subterfuge Discovered}
Taridan discovers the Surinate's message and orders the salvage team to return to the Bartan cruiser immediately. \textit{Will you help Surinate evade detection?} \textbf{Strategy \& Tactics} vs. \textbf{Bartan Salvage Team} (2d8).

\subsection*{Crisis}
\begin{itemize}
	\item \textit{Will you grant Surinate's request for political asylum?} \textbf{Threats:} Taridan sends a team of soldiers to the station to retrieve Surinate by force. The Bartan cruiser attacks the Endeavour to prevent it from interfering.
	\item \textit{Or will you concede to Taridan's demands and allow Surinate to be taken into custody?} \textbf{Threats:} Surinate sabotages the habitat's navigation beacon, posing a major threat to ships throughout the region. Surinate refuses to surrender, going so far as to activate the habitat's self destruct mechanism if necessary to avoid being arrested.
\end{itemize}


\newpage

\subsection*{Characters}
\begin{description}
	\item[Habitat Six (d8):] Antiquated (d6), Idiosyncratic (d8). 
	\item[Surinate (d6):] Asylum-Seeker (d6), Desperate (d6), Communications Technician (d6), Transitioning (d10).
	\item[Taridan (d8):] Captain of the Bartan Cruiser (d8), Aggressive (d6 \textit{Dangerous}), Honorable (d8).
\end{description}

\subsection*{Places}
\begin{description}
	\item[Operations:] The command center of ICF Habitat Six. Viewscreens occupy one entire wall while the rest of the room is filled with rows of workstations.
	\item[Communications Array:] A complex assortment of antennas and receiver dishes. Accessible only via a series of cramped utility corridors.
	\item[Bartan Cruiser:] A modern warship with strange and powerful weapons technology. More than a match for the ICS Endeavour in a fair fight.
\end{description}

\subsection*{Mysteries}
\begin{description}
	\item[Surinate does not want to become a King.]\phantom{a}\\ Bartan Kings are the only members of their species who can reproduce. Bartans who can reproduce are required by law to do so. \textit{ What does The Transition entail? What happens to Bartan Kings after they reproduce?}

	\item[Taridan's actions here could lead to war.]\phantom{a}\\
	The ICF and the Bartan Empire have a long history of friendly relations. \textit{Why are the Bartans being so aggressive? Why is Taridan willing to incite a diplomatic incident over the question of who owns Habitat Six?}
\end{description}

\newpage
%\thispagestyle{plain}

\ClearShipoutPicture

\AddToShipoutPictureBG{
\begin{tikzpicture}[remember picture, overlay]
	\pic () at (current page.center) {starfield};

	\node[endeavour_box, minimum width=12.6cm, minimum height=18.8cm,
           path picture={
           	  \path (path picture bounding box.south) --++ (-0.15cm,0.0cm) coordinate (x) {};
               \node at (x){
                   \includegraphics[width=12.8cm, height=19cm]{Images/city_cover.png}
               };
           }] at (current page.center) {};
\end{tikzpicture}
}
\phantom{a}

\newpage

\ClearShipoutPicture
\AddToShipoutPictureBG{
	\begin{tikzpicture}[remember picture, overlay]
		\pic () at (current page.center) {starfield};
		\node[endeavour_box, minimum width=12.6cm, minimum height=18.8 cm] at (current page.center) {};
	\end{tikzpicture}
}

\section*{Sendiv}
\textit{\textbf{Captain's Log:} We have arrived in a region of space known as the Kerring Expanse. The region itself is uninhabited. It features prominently, however, in the legends of several nearby civilizations as a place of mystery and peril.}

\textit{One legend in particular tells of a pair of violent storms that rend the sky here once every one hundred solar years.  Anecdotal evidence suggests that this legend may be based in fact.  If so, then the next storm is due to occur any time now.}

\subsection*{Arrival}
A flash blinds the Endeavour's main viewscreen and the ship lurches violently to starboard. The ship rocks beneath you and damage reports start coming in from all decks.

As systems come back online, you see a swirling green vortex that grows with alarming speed as it approaches. A mote in the center of the approaching storm glows ever brighter until it outshines even the nearby stars. 

\subsubsection*{Navigating the Maelstrom}
\begin{itemize}
	\item \textit{Will you focus on weathering the storm to minimize the risk to the Endeavour?} \textbf{Operations \& Engineering} vs. \textbf{The Maelstrom}. This is a \textit{Dangerous} challenge. If you fail, the Endeavour is seriously damaged.
	\item \textit{Or will you focus on studying the storm to learn more about what might have caused it?} \textbf{Science \& Medicine} vs. \textbf{The Maelstrom}. This is a \textit{Grueling} challenge. The Endeavour is seriously damaged. If you prevail, you gain a 1d6 advantage die on all challenges vs \textbf{Sendiv}.
\end{itemize}

When the storm abates, sensors report that a new city has appeared floating in the ocean on one of the moons of a spectacularly-ringed gas giant nearby.
\newpage

\subsection*{Trials}
\subsubsection*{Cultural Exchange}
\textbf{Tzaarn, mayor of Sendiv} contacts you and invites the crew of the Endeavour to join the Sendivians for a night of revelry. The highlight of the evening is a talent show featuring amazing acts of skill and artistry. The Sendivians ask you to contribute something to the event. \textit{Will you perform for your hosts?}
\textbf{Leadership \& Negotiation} vs. \textbf{Revelers} (3d6).

\subsubsection*{Effecting Repairs}
The Endeavour may have been damaged in \textbf{The~Maelstrom}. \textit{Can you repair the ship using only what you have aboard?}
\textbf{Operations \& Engineering} vs. \textbf{Storm Damage} (3d8). \textit{Can you use Sendivian technology to augment your own?} \textbf{Science \& Medicine} vs. \textbf{Sendiv}.

\subsubsection*{No Way Out}
The Sendivians discourage any crew members who enter the city from returning to the Endeavour.  Suspicious mechanical failures conspire to prevent the crew from leaving the city. \textit{Can you force Sendiv into a position that reveals why it is trying to keep you here?}
\textbf{Strategy \& Tactics} vs. \textbf{Sendiv}.


\subsection*{Crisis}
\begin{itemize}
	\item \textit{Will you try to escape from the city prior to launch?} \textbf{Threats:} Tzaarn begs you to stay. Sendiv deploys force shields and grappling beams to prevent your escape and discourage outside interference.
	\item \textit{Or will you try to interrupt the launch sequence and prevent the city from continuing its journey?} \\ \textbf{Threats:} Tzaarn refuses to help. Sendiv denies access to critical systems. The Sendivians actively oppose you.
\end{itemize}
\textbf{The Maelstrom} returns when Sendiv departs. Anyone who remains within the city is unharmed but lost forever.

\newpage

\subsection*{Characters}
\begin{description}
	\item[The Maelstrom (d12):] Turbulent (d8), Confusing (d8), Violent (d8 \textit{Dangerous}), Short-Lived (d8).
	\item[Tzaarn (d8):] Mayor of Sendiv (d6), Beautiful (d10), Charming (d8), Intelligent (d8), Athletic (d6).
	\item[Sendiv (d10):] Artificial Intelligence (d10), Powerful (d8), Mysterious (d8), Indifferent (d6).
\end{description}

\subsection*{Places}
\begin{description}
	\item[Auditorium:] Huge and opulent. Tiered seating surrounds a central stage. Windows provide spectacular views of the stars above and the city below.
	\item[Tzaarn's Quarters:] Large and well appointed. The bedroom is dominated by an elaborate stasis chamber.
	\item[Engine Room:] Clean and well-kept. Filled with the sound of machinery hidden behind sealed access panels.
\end{description}

\subsection*{Mysteries}
\begin{description}
	\item[Sendiv is a city-sized spaceship.] \phantom{a} \\ It travels a galaxy-spanning circuit, preserving the Sendivians in stasis while it is underway. Each leg of this journey takes many years. \textit{Does the civilization that built Sendiv still exist? It Sendiv something more than just an extravagant luxury liner?}
	\item[Tzaarn did not intend to trap you on Sendiv.] \phantom{a} \\ Throughout their journey, the Sendivians have welcomed people into their city, shared a night of revelry, and then parted as friends. \textit{Why is Sendiv so reluctant to let you go? Does it need your protection? Is it trying to protect the crew of the Endeavour from some unknown threat?}
\end{description}

\newpage

\section*{The Vault of Heaven}
The Vault of Heaven is what you will use to track your progress over the course of a campaign. During each adventure, your actions can affect the crew's relationships with the Civilizations that you have encountered.

\newlength{\terranlen}
\setlength{\terranlen}{\widthof{\futura{Terran}}}

\newcommand\dunderline[3][-1pt]{{%
  \sbox0{#3}%
  \ooalign{\copy0\cr\rule[\dimexpr#1-#2\relax]{\wd0}{#2}}}}
  
\begin{center}
\begin{tabular}{r@{\qquad}c@{\qquad}c} \toprule
\futura{Civilization} & \futura{Favor} & \futura{Wrath} \\ \midrule \\[-2ex]
\dunderline{\lightrulewidth}{\phantom{Civilization}}\hspace{-\terranlen}\futura{Terran} & \tikz[baseline=-0.75ex]{\pic {blank_divine_favor};}\,\,\,\tikz[baseline=-0.75ex]{\pic {blank_divine_favor};}\,\,\,\tikz[baseline=-0.75ex]{\pic {blank_divine_favor};} & \tikz[baseline=-1.125ex]{\pic {wrath_triangle};}\,\,\,\tikz[baseline=-1.125ex]{\pic {wrath_triangle};}\,\,\,\tikz[baseline=-1.125ex]{\pic {wrath_triangle};}\\[1.5ex]

\dunderline{\lightrulewidth}{\phantom{Civilization}} & \tikz[baseline=-0.75ex]{\pic {blank_divine_favor};}\,\,\,\tikz[baseline=-0.75ex]{\pic {blank_divine_favor};}\,\,\,\tikz[baseline=-0.75ex]{\pic {blank_divine_favor};} & \tikz[baseline=-1.125ex]{\pic {wrath_triangle};}\,\,\,\tikz[baseline=-1.125ex]{\pic {wrath_triangle};}\,\,\,\tikz[baseline=-1.125ex]{\pic {wrath_triangle};}\\[1.5ex]

\dunderline{\lightrulewidth}{\phantom{Civilization}} & \tikz[baseline=-0.75ex]{\pic {blank_divine_favor};}\,\,\,\tikz[baseline=-0.75ex]{\pic {blank_divine_favor};}\,\,\,\tikz[baseline=-0.75ex]{\pic {blank_divine_favor};} & \tikz[baseline=-1.125ex]{\pic {wrath_triangle};}\,\,\,\tikz[baseline=-1.125ex]{\pic {wrath_triangle};}\,\,\,\tikz[baseline=-1.125ex]{\pic {wrath_triangle};}\\[1.5ex]

\dunderline{\lightrulewidth}{\phantom{Civilization}} & \tikz[baseline=-0.75ex]{\pic {blank_divine_favor};}\,\,\,\tikz[baseline=-0.75ex]{\pic {blank_divine_favor};}\,\,\,\tikz[baseline=-0.75ex]{\pic {blank_divine_favor};} & \tikz[baseline=-1.125ex]{\pic {wrath_triangle};}\,\,\,\tikz[baseline=-1.125ex]{\pic {wrath_triangle};}\,\,\,\tikz[baseline=-1.125ex]{\pic {wrath_triangle};}\\[1.5ex]

\dunderline{\lightrulewidth}{\phantom{Civilization}} & \tikz[baseline=-0.75ex]{\pic {blank_divine_favor};}\,\,\,\tikz[baseline=-0.75ex]{\pic {blank_divine_favor};}\,\,\,\tikz[baseline=-0.75ex]{\pic {blank_divine_favor};} & \tikz[baseline=-1.125ex]{\pic {wrath_triangle};}\,\,\,\tikz[baseline=-1.125ex]{\pic {wrath_triangle};}\,\,\,\tikz[baseline=-1.125ex]{\pic {wrath_triangle};}\\[1.5ex]

\dunderline{\lightrulewidth}{\phantom{Civilization}} & \tikz[baseline=-0.75ex]{\pic {blank_divine_favor};}\,\,\,\tikz[baseline=-0.75ex]{\pic {blank_divine_favor};}\,\,\,\tikz[baseline=-0.75ex]{\pic {blank_divine_favor};} & \tikz[baseline=-1.125ex]{\pic {wrath_triangle};}\,\,\,\tikz[baseline=-1.125ex]{\pic {wrath_triangle};}\,\,\,\tikz[baseline=-1.125ex]{\pic {wrath_triangle};}\\[1.5ex]

\dunderline{\lightrulewidth}{\phantom{Civilization}} & \tikz[baseline=-0.75ex]{\pic {blank_divine_favor};}\,\,\,\tikz[baseline=-0.75ex]{\pic {blank_divine_favor};}\,\,\,\tikz[baseline=-0.75ex]{\pic {blank_divine_favor};} & \tikz[baseline=-1.125ex]{\pic {wrath_triangle};}\,\,\,\tikz[baseline=-1.125ex]{\pic {wrath_triangle};}\,\,\,\tikz[baseline=-1.125ex]{\pic {wrath_triangle};}\\[1.5ex]

\dunderline{\lightrulewidth}{\phantom{Civilization}} & \tikz[baseline=-0.75ex]{\pic {blank_divine_favor};}\,\,\,\tikz[baseline=-0.75ex]{\pic {blank_divine_favor};}\,\,\,\tikz[baseline=-0.75ex]{\pic {blank_divine_favor};} & \tikz[baseline=-1.125ex]{\pic {wrath_triangle};}\,\,\,\tikz[baseline=-1.125ex]{\pic {wrath_triangle};}\,\,\,\tikz[baseline=-1.125ex]{\pic {wrath_triangle};}\\[1.5ex]

\dunderline{\lightrulewidth}{\phantom{Civilization}} & \tikz[baseline=-0.75ex]{\pic {blank_divine_favor};}\,\,\,\tikz[baseline=-0.75ex]{\pic {blank_divine_favor};}\,\,\,\tikz[baseline=-0.75ex]{\pic {blank_divine_favor};} & \tikz[baseline=-1.125ex]{\pic {wrath_triangle};}\,\,\,\tikz[baseline=-1.125ex]{\pic {wrath_triangle};}\,\,\,\tikz[baseline=-1.125ex]{\pic {wrath_triangle};}\\[1.5ex]

\dunderline{\lightrulewidth}{\phantom{Civilization}} & \tikz[baseline=-0.75ex]{\pic {blank_divine_favor};}\,\,\,\tikz[baseline=-0.75ex]{\pic {blank_divine_favor};}\,\,\,\tikz[baseline=-0.75ex]{\pic {blank_divine_favor};} & \tikz[baseline=-1.125ex]{\pic {wrath_triangle};}\,\,\,\tikz[baseline=-1.125ex]{\pic {wrath_triangle};}\,\,\,\tikz[baseline=-1.125ex]{\pic {wrath_triangle};}\\[1.5ex]

\dunderline{\lightrulewidth}{\phantom{Civilization}} & \tikz[baseline=-0.75ex]{\pic {blank_divine_favor};}\,\,\,\tikz[baseline=-0.75ex]{\pic {blank_divine_favor};}\,\,\,\tikz[baseline=-0.75ex]{\pic {blank_divine_favor};} & \tikz[baseline=-1.125ex]{\pic {wrath_triangle};}\,\,\,\tikz[baseline=-1.125ex]{\pic {wrath_triangle};}\,\,\,\tikz[baseline=-1.125ex]{\pic {wrath_triangle};}\\[1.5ex]

\dunderline{\lightrulewidth}{\phantom{Civilization}} & \tikz[baseline=-0.75ex]{\pic {blank_divine_favor};}\,\,\,\tikz[baseline=-0.75ex]{\pic {blank_divine_favor};}\,\,\,\tikz[baseline=-0.75ex]{\pic {blank_divine_favor};} & \tikz[baseline=-1.125ex]{\pic {wrath_triangle};}\,\,\,\tikz[baseline=-1.125ex]{\pic {wrath_triangle};}\,\,\,\tikz[baseline=-1.125ex]{\pic {wrath_triangle};}\\[1.0ex] \bottomrule
\end{tabular}
\end{center}

If you do something that a Civilization approves of, the crew will earn a mark of \tikz[baseline=-0.75ex]{\pic {blank_divine_favor};}~(Favor) with that Civilization.  If you do something that a Civilization disapproves of, the crew will earn a mark of \tikz[baseline=-1.125ex]{\pic {wrath_triangle};}~(Wrath) with that Civilization.

\newpage

\thispagestyle{empty}

\tikzset{starfield/.pic={
	\node () at (current page.center) {\includegraphics[width=\pagewidth, height=\pageheight]{Images/starfield.png}};
}}

\ClearShipoutPicture
\AddToShipoutPictureBG{
	\begin{tikzpicture}[remember picture, overlay]
	\pic () at (current page.center) {starfield};
	\end{tikzpicture}
}

\phantom{a}
\end{document}

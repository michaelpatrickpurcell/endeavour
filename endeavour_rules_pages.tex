\documentclass[11pt, a5paper, parskip=half-, DIV=12]{scrartcl}

\usepackage{endeavour}

\colorlet{blackfillcolor}{black}
\colorlet{darkfillcolor}{LCARS_ORANGE}
\colorlet{lightfillcolor}{lightgray!25}
\colorlet{borderfillcolor}{lightgray}
\colorlet{subheadercolor}{lightfillcolor}
\colorlet{starcolor}{LCARS_LIGHTGRAY}
\colorlet{bgcolor}{LCARS_DARKGRAY}
\colorlet{sectioncolor}{darkfillcolor}
\colorlet{headfootcolor}{lightfillcolor}
\pagecolor{bgcolor}

%\colorlet{blackfillcolor}{black}
%\colorlet{darkfillcolor}{LCARS_ORANGE}
%\colorlet{lightfillcolor}{white}
%\colorlet{borderfillcolor}{lightgray}
%\colorlet{subheadercolor}{black}
%\colorlet{starcolor}{white}
%\colorlet{bgcolor}{white}
%\colorlet{sectioncolor}{darkfillcolor}
%\colorlet{headfootcolor}{black}
%\pagecolor{bgcolor}


%\colorlet{blackfillcolor}{black}
%\colorlet{darkfillcolor}{lightgray}
%\colorlet{lightfillcolor}{white}
%\colorlet{borderfillcolor}{lightgray}
%\colorlet{subheadercolor}{black}
%\colorlet{starcolor}{white}
%\colorlet{bgcolor}{white}
%\colorlet{sectioncolor}{black}
%\colorlet{headfootcolor}{black}
%\pagecolor{bgcolor}

\tikzset{starfield/.pic={
	\node () at (current page.center) {\includegraphics[width=\pagewidth, height=\pageheight]{Images/starfield.png}};
}}

\usepackage{scrlayer-scrpage} % Manage headers and footers in Koma-Script document classes

\usepackage[bmargin=1.66cm, lmargin=1.97cm, rmargin=1.97cm, tmargin=1.66cm, footskip=1.33cm]{geometry} % Set margins and footer placement

\usepackage{tikz}
\usetikzlibrary{arrows.meta} % Customize arrowheads in in Write the Outline example
\usetikzlibrary{shapes} % Draw the node in the fancy \pagemark
\usepackage[skins]{tcolorbox}

%\usepackage[skins]{tcolorbox} % Add background image to tikz node in color cover

\usepackage{fontspec} % Use system fonts.  Not compatible with pdflatex. Use XeLaTeX instead!
\setmainfont{TeX Gyre Schola}
\setmonofont{Courier Prime}
\setsansfont{Futura}
\newcommand{\futura}[1]{{\setmainfont{Futura} #1}}

\usepackage{enumitem} % Adjust formatting of description environment items
\usepackage{multicol} % Format list of playtesters in two columns

\usepackage[type={CC}, version={4.0}, modifier={by-sa}]{doclicense} % Add text and icons for creative commons license

\usepackage{eso-pic}

\usepackage{booktabs}
\usepackage{caption}
\usepackage{multirow}
\usepackage{scalerel}
\usepackage{array}
\usepackage{contour}
\contourlength{0.15pt}
\contournumber{256}

\usepackage[hidelinks]{hyperref} % Add hyperlinks to the pdf file. This should usually be the last package loaded before \begin{document}
\usepackage[xspace]{ellipsis} % Properly typeset ellipses. This package is one of the few packages that must be loaded after hyperref


% Define command to set value of a variable that contains the version number
% This is meant to mimic the syntax used for the built-in \title{} and \author{} commands.
\makeatletter
\newcommand{\version}[1]{\newcommand{\@version}{#1}}
\makeatother


% Set header and footer text
\clearpairofpagestyles
%\makeatletter
%\chead*{\setmainfont{Futura} \normalshape \small \colorbox{bgcolor}{\textcolor{headfootcolor}{Version \@version}}}
%\makeatother
%\cfoot{\setmainfont{Futura} \normalshape \small \colorbox{bgcolor}{\textcolor{headfootcolor}{\arabic{page}}}}
\makeatletter
\chead*{\setmainfont{Futura} \normalshape \small {\textcolor{headfootcolor}{Version \@version}}}
\makeatother
\cfoot{\setmainfont{Futura} \normalshape \small {\textcolor{headfootcolor}{\arabic{page}}}}

% Adjust spacing before and after section headings
\RedeclareSectionCommand[
  runin=false,
%  beforeskip=2.0\baselineskip,
  afterskip=0.25\baselineskip
]{section}

% Adjust spacing before and after subsection headings
\RedeclareSectionCommand[
  runin=false,
%  beforeskip=2.0\baselineskip,
  afterskip=0.25\baselineskip
]{subsection}

% Adjust spacing before and after subsubsection headings
\RedeclareSectionCommand[
  runin=false,
  beforeskip=0.0\baselineskip,
  afterskip=-0.4\baselineskip
]{subsubsection}

% Adjust formatting of description environment
%\renewcommand{\descriptionlabel}[1]{%
%  \hspace\labelsep \upshape\bfseries #1:%
%}
\setlist[description]{labelindent=0.25cm, leftmargin=\widthof{\hspace{0.25cm}\textbullet\space}, font=\normalfont\textbullet\bfseries\space}

% Adjust formatting of itemize environment
%itemize environments should be followed immediately by a \vspace{1ex} command.
\setlist[itemize]{labelindent=0.25cm, leftmargin=*, labelsep=\widthof{\space}}

\setkomafont{section}{\color{sectioncolor}\setmainfont{Futura}\LARGE}
\setkomafont{subsection}{\color{sectioncolor}\setmainfont{Futura}\Large}
\setkomafont{subsubsection}{\color{sectioncolor}\setmainfont{Futura}\large}

% Set some environment variables for use in the title page
\title{ENDEAVOUR}
\subtitle{A PARAGON Playset}
\author{Michael Purcell}
\version{0.1}

\newcommand{\tablesep}{\,\,$\Longleftrightarrow$\,\,}
\newcommand{\AGON}{\textsc{Agon}}%{{\setmainfont{Calluna-Black} AGON}}
\newcommand{\ENDEAVOUR}{\textsc{Endeavour}}%{{\setmainfont{TTMussels-BoldItalic} ENDEAVOUR}}

\begin{document}

% Colour Cover
\thispagestyle{plain}
\begin{titlepage}
\begin{tikzpicture}[remember picture, overlay]
	\node () at (current page.center) {\includegraphics[width=\pagewidth, height=\pageheight]{Images/endeavour_cover.png}};
\end{tikzpicture}
\end{titlepage}



\AddToShipoutPictureBG{
	\begin{tikzpicture}[remember picture, overlay]
	\pic () at (current page.center) {starfield};
		\node[endeavour_box, minimum width=12.6cm, minimum height=18.8 cm] at (current page.center) {};
	\end{tikzpicture}
}


\setcounter{page}{1}
\setmainfont{TeX Gyre Schola}
\normalsize
\raggedright

\section*{Premise}
%The year is AFC 143, one hundred forty-three years after first contact.
The year is 2364.
The people of Earth, together with their network of far-flung colonies, comprise the Terran Civilization. The Terran Civilization, in turn, is a member of the Interstellar Confederation, an organization made up of hundreds of spacefaring civilizations and populated by dozens of different species.

You are an officer aboard the Interstellar Confederation Ship (ICS) Endeavour. Your mission is to explore the galaxy. You will travel deep into uncharted space where you will encounter strange natural phenomena, make first contact with alien civilizations, and help those in need.

\subsection*{Influences}
\ENDEAVOUR{} is based on \textsc{Star Trek}. In particular, its design has been informed by the series that Gene Roddenberry was personally involved with: \textsc{Star Trek} (the original series) and \textsc{Star Trek: The Next Generation}.

It has also been informed by other television shows such as \textsc{Quantum Leap}, \textsc{seaQuest DSV}, and \textsc{Stargate SG-1}.%\textsc{Battlestar Galactica}, \textsc{Farscape}, \textsc{Firefly}, and \textsc{The Orville}.

Other major influences include the works of Isaac~Asimov, Arthur~C.~Clarke, Robert~Heinlein, and Carl~Sagan.
 
The ICS Endeavour is named after the HMS Endeavour, the famed British Royal Navy research vessel that Captain Cook commanded during his first voyage of discovery.
\newpage

\section*{Tone}
This is an optimistic-science fiction game. It is a game about a future in which humanity has progressed beyond the kinds of internecine conflicts that plague modern society. Advanced technology is common and has created a post-scarcity society throughout the Interstellar Confederation.

The crew of the ICS Endeavour are skilled professionals who know how to work together effectively. While they may disagree at times, they will not let personal biases or animosity affect their work.

Stories in \ENDEAVOUR{} generally involve some kind of moral quandary. Moreover, the futuristic setting acts as lens through which we can view contemporary social issues. The best such stories are fundamentally about the difficult choices the crew are asked to make and how they are affected by their experiences.  

\subsection*{Technology}
\ENDEAVOUR{} is about people more than it is about technology. That said, there are a few technologies that are commonly assumed to be available in order to make stories about deep-space exploration plausible and relatable.
\begin{description}
	\item[Faster Than Light Travel]
	\item[Artificial Gravity]
	\item[Universal Translation]
\end{description}
Without these technologies, or when these technologies fail, stories tend to focus on the consequences of their absence rather than on the characters.

\newpage

\section*{The Interstellar Confederation}
The Interstellar Confederation is diverse and multicultural society. Its member civilizations lay claim to thousands of star systems and its population numbers in the quadrillions. 

The size and scope of the Interstellar Confederation makes it nearly impossible for any centralized government to manage it all. As such, each member civilization retains a significant amount of autonomy. This has resulted in a complex society with a wide range of social and economic structures.

Members civilizations, however, are united by their belief in the value of education, reason, and individualism. These beliefs are manifest in the Interstellar Confederation's prohibition on interfering with non-spacefaring civilizations and its emphasis on exploration.

\subsection*{Aliens}
Humans are only one of many species that populate the Interstellar Confederation. Many of these species are humanoids. Furthermore, most of these species have similar environmental requirements which allows them to live and work together without the need for adaptive technologies.

Beyond the Interstellar Confederation, however, there is a bewildering array of alien life forms that the crew might encounter in the course of their adventures. These range from simple one-celled organisms to enormous creatures that live their entire lives in the vacuum of space.

\newpage

\section*{The ICS Endeavour}
The ICS Endeavour is a ship designed for deep-space exploration and is capable of operating independently for extended periods of time. By collecting raw materials while underway and using efficient recycling systems, it can produce the air, water, and food needed to sustain the crew for up to one solar year.

The Endeavour has a total complement of about five hundred personnel. This crew is comprised of a multidisciplinary team with expertise in fields such as ship's operations, physical and life sciences, and interstellar diplomacy. It also includes representatives of many species and civilizations.

\subsection*{Captain James Darcy}
The current captain of the Endeavour is James Darcy. He is a human from the Terran Civilization and is forty-three solar years old. This is his first command.

Captain Darcy is one of the Interstellar Confederation Fleet's most highly decorated officers. He is a charismatic leader and a cunning tactician. His past assignments include executive officer of the ICS Discovery, chief of operations for Habitat Three, and junior tactical officer aboard the ICS Adventure.

\begin{description}
	\item[James Darcy (d10):] Captain of the Endeavour (d10), Charismatic (d10), Cunning (d8), Experienced (d8).
\end{description}

In this game, Captain Darcy will be a prominent non-player character. As commanding officer, he will not generally face challenges himself. When he does get personally involved, he will do so by giving a copy of his name die (d10) to an officer who used a bond to call upon him while facing a challenge.

\newpage

%\thispagestyle{empty}
\ClearShipoutPicture

\AddToShipoutPictureBG{
\begin{tikzpicture}[remember picture, overlay]
	\pic () at (current page.center) {starfield};

	\node[endeavour_box, minimum width=12.6cm, minimum height=18.8cm,
           path picture={
           	  \path (path picture bounding box.south) --++ (0.5cm,-0.66cm) coordinate (x) {};
               \node at (path picture bounding box.south){
                   \includegraphics[width=12.8cm, height=19cm]{Images/lem_cover.png}
               };
           }] at (current page.center) {};
\end{tikzpicture}
}
\phantom{a}

\newpage

\ClearShipoutPicture
\AddToShipoutPictureBG{
	\begin{tikzpicture}[remember picture, overlay]
		\pic () at (current page.center) {starfield};
		\node[endeavour_box, minimum width=12.6cm, minimum height=18.8 cm] at (current page.center) {};
	\end{tikzpicture}
}


\section*{Systems}
\ENDEAVOUR{} uses a modified version of the \textsc{Paragon System}, found in the \AGON{} roleplaying game. You will need a copy of the \AGON{} rulebook to play this game. Refer to \href{http://www.agon-rpg.com}{agon-rpg.com} for more information. Changes to the \textsc{Paragon System} as used in \ENDEAVOUR{} are described below.

As with \AGON{}, a pool comprised of polyhedral dice is used to determine the outcome of each challenge. It is best if each player has (at least) \textbf{1d4}, \textbf{3d6}, \textbf{3d8}, \textbf{2d10}, and \textbf{1d12} available for use throughout the game.

\subsection*{Modified Terminology}
While \ENDEAVOUR{} uses the same underlying mechanics as \AGON{}, much of the terminology has been changed to reflect the thematic differences between the two games.%
\small
\begin{center}
\begin{tabular}{r@{}c@{}l@{\hskip 4.5ex}r@{}c@{}l} \toprule
\AGON & \tablesep & \ENDEAVOUR & \AGON & \tablesep & \ENDEAVOUR \\ \midrule
Island & \tablesep & Planet & Contest & \tablesep & Challenge \\
& & & Battle & \tablesep & Crisis \\
Gods & \tablesep & Civilizations & Clash & \tablesep & Confrontation \\
Divine Favor & \tablesep & Assistance & & & \\
& & & Pathos & \tablesep & Stress \\
Epithet & \tablesep & Role & Agony & \tablesep & Distress\\
Lineage & \tablesep & Species & Fate & \tablesep & Experience\\ 
Honored God & \tablesep & Heritage & & \\
& & & Glory & \tablesep & Distinction \\
Exodus & \tablesep & Debriefing & & \\
Great Deeds & \tablesep & Discoveries & Harm & \tablesep & Complications \\
Trophies & \tablesep & Artifacts & Perilous & \tablesep & Dangerous \\
& & & Epic & \tablesep & Gruelling \\
Fellowship & \tablesep & Recreation & Sacred & \tablesep & Sensitive\\ 
Sacrifice & \tablesep & Diplomacy & Mythic & \tablesep & Fraught \\ \bottomrule 
\end{tabular}
\end{center}
\normalsize

\newpage

\subsection*{Domains}
Four Domains represent areas of professional expertise for Intersteller Confederation Fleet (ICF) officers. Each challenge falls into one of the four Domains.
\begin{description}
	\item[Leadership \& Negotiation:] Working with others.\\Used for Challenges that require charisma or empathy.
	\item[Science \& Medicine:] Explaining observed phenomena.\\Used for Challenges that require intelligence or creativity.
	\item[Operations \& Engineering:] Managing logistics.\\Used for Challenges that require discipline or precision.
	\item[Strategy \& Tactics:] Outmaneuvering opponents.\\Used for Challenges that require cunning or misdirection.
\end{description}

\subsection*{Virtues}
Four Virtues represent the core values of the Interstellar Confederation. ICF officers are expected to embody these Virtues at all times.
\begin{description}
	\item[Curiosity:] The desire to learn more about the universe.
	\item[Integrity:] Honesty and personal accountability.
	\item[Fairness:] Impartial and just behavior.
	\item[Courage:] The ability to act despite being afraid.
\end{description}
As you play, you may discover that other virtues better describe the core values of the Interstellar Confederation in your game.  If that happens, feel free to replace the existing Virtues with suitable alternatives.

\newpage

\subsection*{Civilizations}
\ENDEAVOUR{} uses Civilizations instead of Gods.  Civilizations include both members of the Interstellar Federation and any other spacefaring peoples that the crew encounter during their adventures.

Only the Terran Civilization is a part of every game. Other Civilizations will be added as they are encountered during play. Whenever you encounter a new Civilization, you should add it to the list of Civilizations on your character sheet.

\subsection*{Discoveries}
\ENDEAVOUR{} uses Discoveries instead of Great Deeds. Discoveries can take many forms.  Some are academic in nature while other are more personal.

Note that if the crew encounter spacefaring aliens who have their own language and culture, then that should be recorded a new Civilization rather than as a Discovery.

\subsection*{Artifacts}
\ENDEAVOUR{} uses Artifacts instead of Trophies. Artifacts are similar to Discoveries. Most Artifacts are physical objects that either house unusual technology or are imbued with cultural significance.

Note that claiming Artifacts can be problematic. Most will be owned by someone else when they are ``found'' by the crew.  These owners are unlikely to allow the crew to take (or keep) Artifacts without raising some kind of objection.

\newpage

\subsection*{Distinction}
\ENDEAVOUR{} uses Distinction instead of Glory. The way in which an officer earns points of Distinction in \ENDEAVOUR{} differs slightly from the way heroes earn Glory in \AGON{}.

When you are best in a Challenge, you earn one point of Distinction. Each time you earn eight points of Distinction, you advance your Name die.

\subsection*{Complications}
\ENDEAVOUR{} uses Complications instead of Harm.
\begin{description}
	\item[Dangerous:] Mark \tikz[baseline=-0.75ex, scale=0.85, transform shape]{\pic {stress_circle};} (Stress) if you suffer.
	\item[Grueling:] Mark \tikz[baseline=-0.75ex, scale=0.85, transform shape]{\pic {stress_circle};} (Stress) to face the Challenge.
	\item[Sensitive:] Spend \tikz[baseline=-0.75ex]{\pic {earned_divine_favor};} (Assistance) if you suffer.
	\item[Fraught:] Spend \tikz[baseline=-0.75ex]{\pic {earned_divine_favor};} (Assistance) to face the Challenge.
\end{description}

\subsection*{Log Entry}
\ENDEAVOUR{} does not use The Signs of the Gods. Instead, each adventure is prefaced by a Log Entry made by one of the crew. The Log Entry serves to establish the setting and provide background information to the players.

Most often, the Log Entry is taken from the captain's log.  This should be prepared in advance and read aloud at the start of the adventure.

The Log Entry should describe where the adventure will take place, what the crew hope to accomplish, and the major characters they expect to encounter.

\newpage

\subsection*{Recreation}
\ENDEAVOUR{} uses Recreation instead of Fellowship. The officers spend time together while off duty. They often pursue mutual hobbies such as playing music or participating in interactive holo-dramas. As they do so, they get to know one another better. 

Taking turns, each player asks a question of another player's character. That player should describe an activity that the two characters do together while off duty and then answer the question. Both players then take a Bond with each other's characters. 

\subsection*{Diplomacy}
\ENDEAVOUR{} uses Diplomacy instead of Sacrifice. The officer who has earned the most points of Distinction leads an effort to strengthen ties between Civilizations. This usually takes the form of informal activities such as athletic competitions, academic conferences, or cultural exchanges.

Diplomacy challenges are always Leadership \& Negotiation challenges.  Everyone who participates marks two \tikz[baseline=-0.75ex]{\pic {earned_divine_favor};}~(Assistance) with Civilizations of their choice. If you are best you also earn a Bond with a Civilization of your choice.

\subsection*{Leadership}
In \ENDEAVOUR{}, the captain is always the leader of the crew. Leadership challenges are used to determine which officer most impresses the captain during each Voyage.

If you are best during a Leadership challenge then you receive a Bond with Captain Darcy. You can use this Bond as usual to ask the captain to Bolster You, Block Complications for You, or Follow Your Lead.

\newpage

%\thispagestyle{empty}
\ClearShipoutPicture

\AddToShipoutPictureBG{
\begin{tikzpicture}[remember picture, overlay]
	\pic () at (current page.center) {starfield};

	\node[endeavour_box, minimum width=12.6cm, minimum height=18.8cm,
           path picture={
           	  \path (path picture bounding box.south) --++ (0.5cm,-0.66cm) coordinate (x) {};
               \node at (path picture bounding box.south){
                   \includegraphics[width=12.8cm, height=19cm]{Images/sunrise_cover.png}
               };
           }] at (current page.center) {};
\end{tikzpicture}
}
\phantom{a}

\newpage

\ClearShipoutPicture

\AddToShipoutPictureBG{
	\begin{tikzpicture}[remember picture, overlay]
		\pic () at (current page.center) {starfield};
		\node[endeavour_box, minimum width=12.6cm, minimum height=18.8 cm] at (current page.center) {};
	\end{tikzpicture}
}


\section*{Character Creation}
Most player characters in \ENDEAVOUR{} are ICF officers who are serving aboard the ICS Endeavour. To create a character, you will:
\begin{enumerate}
	\item Record your Role. This should describe what you do aboard the ICS Endeavour. Common choices include Executive Officer, Chief Engineer, Helmsman, etc.  Your Role die begins at \textbf{d6}.
	\item Record your Name. Your Name die begins at \textbf{d6}.
	\item Record your Species. You may be an alien. If you are, describe your Species to the other players.
	\item Record your Heritage. This is your home Civilization. Record two marks of \tikz[baseline=-0.75ex]{\pic {earned_divine_favor};}~(Assitance) with your home Civilization. Record three more marks of \tikz[baseline=-0.75ex]{\pic {earned_divine_favor};}~(Assitance) with Civilizations of your choice.
	\item Choose a favored Domain. Your favored-Domain die begins at \textbf{d8}. Your other three Domain dice begin at \textbf{d6}.
	\item Record two Bonds with each other player character.
\end{enumerate}
You should work together with the other players to create a diverse group of characters.  This will help to ensure that you are prepared to face whatever adventures await you.

\subsection*{Experienced Characters}
Optionally, you can create an experienced character. To do so, follow the process described above. Then, advance your Experience track as far as you like. Take Boons as usual.

Remember that while each Boon will make you a more capable officer, each \tikz[baseline=-0.75ex, scale=0.85, transform shape]{\pic {experience_box};} (Experience) you mark will bring you one step closer to the end of your tour of duty.

\newpage

\ClearShipoutPicture
\AddToShipoutPictureBG{
	\begin{tikzpicture}[remember picture, overlay]
	\pic () at (current page.center) {starfield};
	\end{tikzpicture}
}

{%
\thispagestyle{empty}
\setmainfont{Futura}
\Large
\begin{tikzpicture}[overlay, remember picture]
\pic () at (current page.north west) {character_sheet_1};
\end{tikzpicture}
}

\newpage
{%
\thispagestyle{empty}
\setmainfont{Futura}
\Large
\begin{tikzpicture}[overlay, remember picture]
\pic () at (current page.north west) {character_sheet_2};
\end{tikzpicture}
}

\newpage
%\ClearShipoutPicture
%\AddToShipoutPictureBG{
%	\begin{tikzpicture}[remember picture, overlay]
%		\pic () at (current page.center) {starfield};
%		\node[endeavour_box, minimum width=12.6cm, minimum height=18.8 cm] at (current page.center) {};
%	\end{tikzpicture}
%}

%\section*{Planets}
%\begin{description}
%	\item[ICF Habitat Six:] Reproductive rights, political asylum.
%	\item[Planet Name:] Right to privacy, cloaking technology.
%	\item[Planet Name:] Pets, cultural norms, death penalty.
%	\item[Planet Name:] Communicable diseases, quarantine.
%	\item[Planet Name:] Exotic life forms, colonialism.
%	\item[Planet Name:] Sabotage, reliance on technology.
%	\item[Planet Name:] Adversarial machine learning.
%\end{description}
%
%\newpage

% Full-page image
%\thispagestyle{empty}
%\ClearShipoutPicture
%\begin{tikzpicture}[remember picture, overlay]
%	\node () at (current page.center) {\includegraphics[width=\pagewidth, height=\pageheight]{Images/habitat_six_cover.png}};
%\end{tikzpicture}
%

% Inset image
%\thispagestyle{empty}
\ClearShipoutPicture

\AddToShipoutPictureBG{
\begin{tikzpicture}[remember picture, overlay]
	\pic () at (current page.center) {starfield};

	\node[endeavour_box, minimum width=12.6cm, minimum height=18.8cm,
           path picture={
           	  \path (path picture bounding box.south) --++ (0.5cm,-0.66cm) coordinate (x) {};
               \node at (path picture bounding box.south){
                   \includegraphics[width=12.8cm, height=19cm]{Images/habitat_six_cover.png}
               };
           }] at (current page.center) {};
\end{tikzpicture}
}
\phantom{a}

\newpage

\ClearShipoutPicture
\AddToShipoutPictureBG{
	\begin{tikzpicture}[remember picture, overlay]
		\pic () at (current page.center) {starfield};
		\node[endeavour_box, minimum width=12.6cm, minimum height=18.8 cm] at (current page.center) {};
	\end{tikzpicture}
}

\section*{ICF Habitat Six}
\textit{\textbf{Captain's Log:} We have arrived at ICF Habitat Six. As one of the ICF's first deep-space habitats, this space station is a relic from another age.}

\textit{Years ago, it was decommissioned and ever since has served as an unmanned navigational beacon. Recently, however, the station stopped relaying telemetry. Sensor readings indicate that the habitat's life support systems have been reactivated.}

\textit{We have been unable to establish communications with whoever might be aboard the station. I have dispatched a landing party to investigate.}

\subsection*{Arrival}
When you arrive at the habitat, you are greeted by a small team of \textbf{Bartan} technicians. They explain that they rely on the telemetry produced by the station and came to investigate the cause of its disruption.

``Thank goodness you're here!'' they say, ``We were starting to worry that the ICF had forgotten about this place. There seems to be a problem with the communications array. Mind if we work together to figure out what's wrong with it?''

\subsubsection*{Covert Communication}
\begin{itemize}[topsep=0ex, partopsep=0ex]
	\item \textit{Will you try to fix the communications array?}\\ \textbf{Operations \& Engineering} vs. \textbf{Habitat Six}.
\end{itemize}
Regardless of the outcome of the challenge, you discover a message hidden in the log files.% It reads "Attention ICF personnel: I am a member of the salvage team currently aboard this station. I have recently begun \textbf{The~Transition} and I do not want to become a King. As such, this message constitutes my formal written request for political asylum."

\textit{Attention ICF personel: I am a member of the Bartan salvage team currently aboard the station. I have recently begun \textbf{The~Transition} and I do not want to become a King. I am formally requesting political asylum. Please help me.}

\newpage

\subsection*{Trials}
\subsubsection*{Finding the Asylum-Seeker}
\textbf{Surinate}, the member of the Bartan salvage team who requested asylum, was careful to conceal his identity. \textit{Can you convince Surinate to reveal himself?} \textbf{Leadership \& Negotiation} vs. \textbf{Surinate}. \textit{Or will you try to identify Surinate by analyzing his behavior?}
\textbf{Science \& Medicine} vs. \textbf{Behavioral Cues} (3d6).

\subsubsection*{Hostile Reinforcements Arrive}
A Bartan cruiser arrives at the station.  Its captain, \textbf{Taridan}, claims ownership of ICF Habitat Six, citing a treaty that classifies any inoperable and unattended spacecraft as salvage. \textit{Can you convince Taridan that his claim is illegitimate?} \textbf{Leadership \& Negotiation} vs. \textbf{Taridan}.

\subsubsection*{Subterfuge Discovered}
Taridan discovers the Surinate's message and orders the salvage team to return to the Bartan cruiser immediately. \textit{Will you help Surinate evade detection?} \textbf{Strategy \& Tactics} vs. \textbf{Bartan Salvage Team} (2d8).

\subsection*{Crisis}
\begin{itemize}
	\item \textit{Will you grant Surinate's request for political asylum?} \textbf{Threats:} Taridan sends a team of soldiers to the station to retrieve Surinate by force. The Bartan cruiser attacks the Endeavour to prevent it from interfering.
	\item \textit{Or will you concede to Taridan's demands and allow Surinate to be taken into custody?} \textbf{Threats:} Surinate sabotages the habitat's navigation beacon, posing a major threat to ships throughout the region. Surinate refuses to surrender, going so far as to activate the habitat's self destruct mechanism if necessary to avoid being arrested.
\end{itemize}


\newpage

\subsection*{Characters}
\begin{description}
	\item[Habitat Six (d8):] Antiquated (d6), Idiosyncratic (d8). 
	\item[Surinate (d6):] Asylum-Seeker (d6), Desperate (d6), Communications Technician (d6), Transitioning (d10).
	\item[Taridan (d8):] Captain of the Baltan Cruiser (d8), Aggressive (d6 \textit{Dangerous}), Honorable (d8).
\end{description}

\subsection*{Places}
\begin{description}
	\item[Operations:] The command center of ICF Habitat Six. Viewscreens occupy one entire wall while the rest of the room is filled with rows of workstations.
	\item[Communications Array:] A complex assortment of antennas and receiver dishes. Accessible only via a series of cramped utility corridors.
	\item[Bartan Cruiser:] A modern warship with strange and powerful weapons technology. More than a match for the ICS Endeavour in a fair fight.
\end{description}

\subsection*{Mysteries}
\begin{description}
	\item[Surinate does not want to become a King.]\phantom{a}\\ Bartan Kings are the only members of their species who can reproduce. Very few Bartans can become Kings, and any who are able are required by law to do so. \textit{ What does The Transition entail? What happens to Bartan Kings after they reproduce?}

	\item[Taridan's actions here could lead to war.]\phantom{a}\\
	The ICF and the Bartan Empire have a long history of friendly relations. \textit{Why are the Bartans being so aggressive? Why is Taridan willing to incite a diplomatic incident over the question of who owns Habitat Six?}
\end{description}

\newpage

%\thispagestyle{empty}
\ClearShipoutPicture

\AddToShipoutPictureBG{
\begin{tikzpicture}[remember picture, overlay]
	\pic () at (current page.center) {starfield};

	\node[endeavour_box, minimum width=12.6cm, minimum height=18.8cm,
           path picture={
           	  \path (path picture bounding box.south) --++ (0.5cm,-0.66cm) coordinate (x) {};
               \node at (path picture bounding box.south){
                   \includegraphics[width=12.8cm, height=19cm]{Images/meteor_shower_cover.png}
               };
           }] at (current page.center) {};
\end{tikzpicture}
}
\phantom{a}

\newpage

\ClearShipoutPicture

\AddToShipoutPictureBG{
	\begin{tikzpicture}[remember picture, overlay]
		\pic () at (current page.center) {starfield};
		\node[endeavour_box, minimum width=12.6cm, minimum height=18.8 cm] at (current page.center) {};
	\end{tikzpicture}
}

\section*{Planet Template}
Brief description of the planet.

\textit{\textbf{Log entry:} Get the crew set up for the Arrival challenge.}

\subsection*{Arrival}
A first challenge that introduces the planet's Strife.

\subsubsection*{Title for the Arrival Challenge}
\begin{itemize}
	\item \textit{Will you ... ?} Test
	\item \textit{Or will you ... ?} Test
\end{itemize}

\newpage

\subsection*{Trials}
Some situations and challenges that drive the action towards the Crisis.

\subsubsection*{Name of the first Trial}
test

\subsubsection*{Name of the second Trial}
test

\subsubsection*{Name of the third Trial}
test

\subsection*{Crisis}
Brief description of the planet's Crisis.
\begin{itemize}
	\item \textit{Will you ... ?} Describe some challenges and associated \textbf{Threats}.
	\item \textit{Or will you ... ?} Describe some challenges and associated \textbf{Threats}.
\end{itemize}

\newpage

\subsection*{Characters}
\begin{description}
	\item[Character Name (d?):] Epithets and associated epithet die sizes.
	\item[Character Name (d?):] Epithets and associated epithet die sizes.
	\item[Character Name (d?):] Epithets and associated epithet die sizes.
\end{description}

\subsection*{Places}
\begin{description}
	\item[Place Name:] Brief description of the place.
	\item[Place Name:] Brief description of the place.
\end{description}

\subsection*{Mysteries}
\begin{description}
	\item[A fact about what is happening on the planet.] \phantom{a} \\ \textit{A leading question about what might be going on.}
	\item[A fact about what is happening on the planet.] \phantom{a} \\ \textit{A leading question about what might be going on.}
\end{description}

\newpage

%\thispagestyle{empty}
\ClearShipoutPicture

\AddToShipoutPictureBG{
\begin{tikzpicture}[remember picture, overlay]
	\pic () at (current page.center) {starfield};

	\node[endeavour_box, minimum width=12.6cm, minimum height=18.8cm,
           path picture={
           	  \path (path picture bounding box.south) --++ (-0.15cm,0.0cm) coordinate (x) {};
               \node at (x){
                   \includegraphics[width=12.8cm, height=19cm]{Images/city_cover.png}
               };
           }] at (current page.center) {};
\end{tikzpicture}
}
\phantom{a}

\newpage

\ClearShipoutPicture
\AddToShipoutPictureBG{
	\begin{tikzpicture}[remember picture, overlay]
		\pic () at (current page.center) {starfield};
		\node[endeavour_box, minimum width=12.6cm, minimum height=18.8 cm] at (current page.center) {};
	\end{tikzpicture}
}

\section*{Planet Template}
Brief description of the planet.

\textit{\textbf{Log entry:} Get the crew set up for the Arrival challenge.}

\subsection*{Arrival}
A first challenge that introduces the planet's Strife.

\subsubsection*{Title for the Arrival Challenge}
\begin{itemize}
	\item \textit{Will you ... ?} Test
	\item \textit{Or will you ... ?} Test
\end{itemize}

\newpage

\subsection*{Trials}
Some situations and challenges that drive the action towards the Crisis.

\subsubsection*{Name of the first Trial}
test

\subsubsection*{Name of the second Trial}
test

\subsubsection*{Name of the third Trial}
test

\subsection*{Crisis}
Brief description of the planet's Crisis.
\begin{itemize}
	\item \textit{Will you ... ?} Describe some challenges and associated \textbf{Threats}.
	\item \textit{Or will you ... ?} Describe some challenges and associated \textbf{Threats}.
\end{itemize}

\newpage

\subsection*{Characters}
\begin{description}
	\item[Character Name (d?):] Epithets and associated epithet die sizes.
	\item[Character Name (d?):] Epithets and associated epithet die sizes.
	\item[Character Name (d?):] Epithets and associated epithet die sizes.
\end{description}

\subsection*{Places}
\begin{description}
	\item[Place Name:] Brief description of the place.
	\item[Place Name:] Brief description of the place.
\end{description}

\subsection*{Mysteries}
\begin{description}
	\item[A fact about what is happening on the planet.] \phantom{a} \\ \textit{A leading question about what might be going on.}
	\item[A fact about what is happening on the planet.] \phantom{a} \\ \textit{A leading question about what might be going on.}
\end{description}

\newpage

\section*{The Vault of Heaven}
The Vault of Heaven is what you will use to track your progress over the course of a campaign. During each adventure, your actions can affect the crew's relationships with the Civilizations that you have encountered.

\newlength{\terranlen}
\setlength{\terranlen}{\widthof{\futura{Terran}}}

\newcommand\dunderline[3][-1pt]{{%
  \sbox0{#3}%
  \ooalign{\copy0\cr\rule[\dimexpr#1-#2\relax]{\wd0}{#2}}}}
  
\begin{center}
\begin{tabular}{r@{\qquad}c@{\qquad}c} \toprule
\futura{Civilization} & \futura{Favor} & \futura{Wrath} \\ \midrule \\[-2ex]
\dunderline{\lightrulewidth}{\phantom{Civilization}}\hspace{-\terranlen}\futura{Terran} & \tikz[baseline=-0.75ex]{\pic {blank_divine_favor};}\,\,\,\tikz[baseline=-0.75ex]{\pic {blank_divine_favor};}\,\,\,\tikz[baseline=-0.75ex]{\pic {blank_divine_favor};} & \tikz[baseline=-1.125ex]{\pic {wrath_triangle};}\,\,\,\tikz[baseline=-1.125ex]{\pic {wrath_triangle};}\,\,\,\tikz[baseline=-1.125ex]{\pic {wrath_triangle};}\\[1.5ex]

\dunderline{\lightrulewidth}{\phantom{Civilization}} & \tikz[baseline=-0.75ex]{\pic {blank_divine_favor};}\,\,\,\tikz[baseline=-0.75ex]{\pic {blank_divine_favor};}\,\,\,\tikz[baseline=-0.75ex]{\pic {blank_divine_favor};} & \tikz[baseline=-1.125ex]{\pic {wrath_triangle};}\,\,\,\tikz[baseline=-1.125ex]{\pic {wrath_triangle};}\,\,\,\tikz[baseline=-1.125ex]{\pic {wrath_triangle};}\\[1.5ex]

\dunderline{\lightrulewidth}{\phantom{Civilization}} & \tikz[baseline=-0.75ex]{\pic {blank_divine_favor};}\,\,\,\tikz[baseline=-0.75ex]{\pic {blank_divine_favor};}\,\,\,\tikz[baseline=-0.75ex]{\pic {blank_divine_favor};} & \tikz[baseline=-1.125ex]{\pic {wrath_triangle};}\,\,\,\tikz[baseline=-1.125ex]{\pic {wrath_triangle};}\,\,\,\tikz[baseline=-1.125ex]{\pic {wrath_triangle};}\\[1.5ex]

\dunderline{\lightrulewidth}{\phantom{Civilization}} & \tikz[baseline=-0.75ex]{\pic {blank_divine_favor};}\,\,\,\tikz[baseline=-0.75ex]{\pic {blank_divine_favor};}\,\,\,\tikz[baseline=-0.75ex]{\pic {blank_divine_favor};} & \tikz[baseline=-1.125ex]{\pic {wrath_triangle};}\,\,\,\tikz[baseline=-1.125ex]{\pic {wrath_triangle};}\,\,\,\tikz[baseline=-1.125ex]{\pic {wrath_triangle};}\\[1.5ex]

\dunderline{\lightrulewidth}{\phantom{Civilization}} & \tikz[baseline=-0.75ex]{\pic {blank_divine_favor};}\,\,\,\tikz[baseline=-0.75ex]{\pic {blank_divine_favor};}\,\,\,\tikz[baseline=-0.75ex]{\pic {blank_divine_favor};} & \tikz[baseline=-1.125ex]{\pic {wrath_triangle};}\,\,\,\tikz[baseline=-1.125ex]{\pic {wrath_triangle};}\,\,\,\tikz[baseline=-1.125ex]{\pic {wrath_triangle};}\\[1.5ex]

\dunderline{\lightrulewidth}{\phantom{Civilization}} & \tikz[baseline=-0.75ex]{\pic {blank_divine_favor};}\,\,\,\tikz[baseline=-0.75ex]{\pic {blank_divine_favor};}\,\,\,\tikz[baseline=-0.75ex]{\pic {blank_divine_favor};} & \tikz[baseline=-1.125ex]{\pic {wrath_triangle};}\,\,\,\tikz[baseline=-1.125ex]{\pic {wrath_triangle};}\,\,\,\tikz[baseline=-1.125ex]{\pic {wrath_triangle};}\\[1.5ex]

\dunderline{\lightrulewidth}{\phantom{Civilization}} & \tikz[baseline=-0.75ex]{\pic {blank_divine_favor};}\,\,\,\tikz[baseline=-0.75ex]{\pic {blank_divine_favor};}\,\,\,\tikz[baseline=-0.75ex]{\pic {blank_divine_favor};} & \tikz[baseline=-1.125ex]{\pic {wrath_triangle};}\,\,\,\tikz[baseline=-1.125ex]{\pic {wrath_triangle};}\,\,\,\tikz[baseline=-1.125ex]{\pic {wrath_triangle};}\\[1.5ex]

\dunderline{\lightrulewidth}{\phantom{Civilization}} & \tikz[baseline=-0.75ex]{\pic {blank_divine_favor};}\,\,\,\tikz[baseline=-0.75ex]{\pic {blank_divine_favor};}\,\,\,\tikz[baseline=-0.75ex]{\pic {blank_divine_favor};} & \tikz[baseline=-1.125ex]{\pic {wrath_triangle};}\,\,\,\tikz[baseline=-1.125ex]{\pic {wrath_triangle};}\,\,\,\tikz[baseline=-1.125ex]{\pic {wrath_triangle};}\\[1.5ex]

\dunderline{\lightrulewidth}{\phantom{Civilization}} & \tikz[baseline=-0.75ex]{\pic {blank_divine_favor};}\,\,\,\tikz[baseline=-0.75ex]{\pic {blank_divine_favor};}\,\,\,\tikz[baseline=-0.75ex]{\pic {blank_divine_favor};} & \tikz[baseline=-1.125ex]{\pic {wrath_triangle};}\,\,\,\tikz[baseline=-1.125ex]{\pic {wrath_triangle};}\,\,\,\tikz[baseline=-1.125ex]{\pic {wrath_triangle};}\\[1.5ex]

\dunderline{\lightrulewidth}{\phantom{Civilization}} & \tikz[baseline=-0.75ex]{\pic {blank_divine_favor};}\,\,\,\tikz[baseline=-0.75ex]{\pic {blank_divine_favor};}\,\,\,\tikz[baseline=-0.75ex]{\pic {blank_divine_favor};} & \tikz[baseline=-1.125ex]{\pic {wrath_triangle};}\,\,\,\tikz[baseline=-1.125ex]{\pic {wrath_triangle};}\,\,\,\tikz[baseline=-1.125ex]{\pic {wrath_triangle};}\\[1.5ex]

\dunderline{\lightrulewidth}{\phantom{Civilization}} & \tikz[baseline=-0.75ex]{\pic {blank_divine_favor};}\,\,\,\tikz[baseline=-0.75ex]{\pic {blank_divine_favor};}\,\,\,\tikz[baseline=-0.75ex]{\pic {blank_divine_favor};} & \tikz[baseline=-1.125ex]{\pic {wrath_triangle};}\,\,\,\tikz[baseline=-1.125ex]{\pic {wrath_triangle};}\,\,\,\tikz[baseline=-1.125ex]{\pic {wrath_triangle};}\\[1.5ex]

\dunderline{\lightrulewidth}{\phantom{Civilization}} & \tikz[baseline=-0.75ex]{\pic {blank_divine_favor};}\,\,\,\tikz[baseline=-0.75ex]{\pic {blank_divine_favor};}\,\,\,\tikz[baseline=-0.75ex]{\pic {blank_divine_favor};} & \tikz[baseline=-1.125ex]{\pic {wrath_triangle};}\,\,\,\tikz[baseline=-1.125ex]{\pic {wrath_triangle};}\,\,\,\tikz[baseline=-1.125ex]{\pic {wrath_triangle};}\\[1.0ex] \bottomrule
\end{tabular}
\end{center}

If you do something that a Civilization approves of, the crew will earn a mark of \tikz[baseline=-0.75ex]{\pic {blank_divine_favor};}~(Favor) with that Civilization.  If you do something that a Civilization disapproves of, the crew will earn a mark of \tikz[baseline=-1.125ex]{\pic {wrath_triangle};}~(Wrath) with that Civilization.

\newpage
\section*{Acknowlegements}
Much of the look and feel of \ENDEAVOUR{} is derived from its art, all of which was created by \textbf{svekloid}. This art was assembled from multiple collections available online at \href{http://shutterstock.com}{shutterstock.com} and then modified by Michael Purcell.  

\subsection*{Playtesters} \label{subsection:playtesters}
The following people helped to create \ENDEAVOUR{} by playing early versions of the game and providing invaluable feedback.\vspace{-1.75ex}
\begin{multicols}{2}
\begin{itemize}[noitemsep]
  \item Keydan Bruce
  \item Farzana Choudhury
  \item Michael Cromer
  \item Dannielle Harden
  \item Andrew Hellyer
  \item Sarah Hewat
  \item Scott Joblin
  \item David McKenzie
  \item Holly Moore
  \item Paul Murray
  \item Kira Purcell
  \item Luke Purcell
  \item Steve Purcell
  \item Jo Stephenson
  \item Brett Witty
  \item Bevis Worchester
\end{itemize}
\end{multicols}

\subsection*{Design Tools} \label{subsection:design-tools}
The following tools were used to create this rule book:
\begin{description}[font=\normalfont\textbullet\space, noitemsep, topsep=-1ex]
	\item[XeLaTeX:] Typesetting and layout.
	\item[TikZ:] Diagrams and art.
\end{description}
\vspace{1ex}
The fonts used are {\setmainfont{TT Mussels-BoldItalic} TT~Mussels~Bold~Italic},  \textsf{Futura}, and TeX~Gyre~Schola (cf. Century Schoolbook).

\vfill

\begin{tabular}{@{}m{7.775cm}@{\hspace*{0.375cm}}>{\centering\arraybackslash}m{2.6cm}@{}}
\textbf{Contact:} \href{mailto:endeavour.ttrpg@gmail.com}{endeavour.ttrpg@gmail.com}\newline \phantom{This is a test, only a test.} \newline \footnotesize{For use with the \textsc{Paragon} system, ©2020\newline \textbf{John Harper \& Sean Nittner}. \href{http://agon-rpg.com}{AGON-RPG.com}} & \includegraphics[scale=0.175]{Images/paragon_logo_mark.png} \\[5ex]
{\footnotesize{\doclicenseLongText}} & \Huge{\doclicenseIcon}
\end{tabular}

\end{document}
